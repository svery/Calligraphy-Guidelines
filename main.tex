%%%%%%%%%%%%%%%%%%%%
% Guideline generator for calligraphy
% Original author: svery (reddit.com/u/svery)
% License: CC BY-NC-SA 4.0 (http://creativecommons.org/licenses/by-nc-sa/4.0/)
%%%%%%%%%%%%%%%%%%%%

% paper configurations
\documentclass[letterpaper]{article}
% change to a4paper for a4 paper, etc
% add landscape (say \documentclass[letterpaper, landscape]{article}) for landscape orientation
\usepackage[top=1cm, bottom=1cm, left=1cm, right=1cm]{geometry}
% change margins as you please


%calculation and graphing packages
\usepackage{calc, tikz, ifthen}
\usepackage[nomessages]{fp}
\usetikzlibrary{calc, intersections}


% input 
% haven't figured out how to adapt to different units without having to change a bunch of places yet, sorry
\def\pagewidth{195} % width of guidelines section in mms, no wider than the paper
\def\pagelength{250} % length of guidelines section in mms, no longer than the paper
\def\nibwidth{1.1mm} % nib width in mms, for pointed pen just enter x-height and fill in proportions below, or put 1mm in nibwidth and put the ascenders, etc in number of mms below
\def\ascender{4} % ascender in nib widths
\def\xheight{5} % x-height in nib widths
\def\descender{4} % descender in nib widths
\def\linespace{2} % space between lines in nib widths
%
% line weight, adjust for thicker / thinner lines
\def\lineweight{0.2pt}
%
% slant
\newboolean{ifslant}
\setboolean{ifslant}{true} % change to false if not needed
\def\slant{85} % slant from horizontal. eg. spencerian would be 52 and italics would be in the 80s
\def\slantfreq{10} % length between each slant line in mms
\begin{document}
	% calculations, ignore--I just couldn't figure out how to do this within the \draw command
	\FPeval{\ascx}{\ascender + \xheight}
	\FPeval{\letterh}{\ascx+\descender}
	\FPeval{\lineheight}{\letterh+\linespace}
	%
	% slant calculations, do not need if ifslant is set to false
	\ifthenelse{\boolean{ifslant}}{
		\FPeval{\slantradians}{(\FPpi*\slant) / \pagewidth}
		\FPeval{\slantbegin}{\pagelength / tan \slantradians}
		\FPeval{\slantend}{(\slantbegin+\pagelength) / \slantfreq}
		}{}
	%
	% gets rid of page numbering because I love everything else about the article class
	% if you want the page number, just delete this line or change 'gobble' to 'arabic', 'roman', etc
	\pagenumbering{gobble}
	%
	% title line
	\centering Italic \ascender-\xheight-\descender, nib width \nibwidth \\
	% insert whatever title you want, or delete this line if you want a clean page of guidelines
	%
	% actual plotting
	\begin{tikzpicture}[x=1mm, y=1mm]
	% clip to reduce mess
	\coordinate (A) at (0, 0);
	\coordinate (B) at (\pagewidth, 0);
	\coordinate (C) at (0, -\pagelength);
	\coordinate (D) at (\pagewidth, -\pagelength);
	\clip (A) -- (B) -- (D) -- (C) -- (A);
	
	% horizontal lines
	\foreach \x in {0,...,100}{
		\pgfmathsetmacro\result{-\x * \lineheight * \nibwidth}
		\foreach \length in {0, \ascender, \ascx, \letterh, \lineheight}{
			\draw [line width=\lineweight] ([yshift=\result] (0, -\length*\nibwidth ) -- +(\pagewidth, 0);
			% can change line style by adding options such as [line width=\lineweight, dotted, red], etc (if you want different line settings for different lines just expand the for loop)
		}
	}
	
	% slant lines, will not show if ifslant is set to false
	\ifthenelse{\boolean{ifslant}}{
		\foreach \x in {0,...,\slantend}{
			\pgfmathsetmacro\result{\x * \slantfreq mm}
			\draw [line width=\lineweight] ([xshift=\result] (-\slantbegin, -\pagelength) -- +(\slant:400);}
		}{}
	\end{tikzpicture}
\end{document}
