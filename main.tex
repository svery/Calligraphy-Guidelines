%%%%%%%%%%%%%%%%%%%%
% Guideline generator for calligraphy
% Original author: svery (reddit.com/u/svery)
% License: CC BY-NC-SA 4.0 (http://creativecommons.org/licenses/by-nc-sa/4.0/)
% To use on Overleaf: Click the Preview button to the right, download by clicking the PDF button on top
%%%%%%%%%%%%%%%%%%%%

% paper configurations
\documentclass[letterpaper]{article}
% change to a4paper for a4 paper, etc
% add landscape (say \documentclass[letterpaper, landscape]{article}) for landscape orientation
% if you want other paper size, delete the letterpaper option and go to the next line
\usepackage[top=1cm, bottom=1cm, left=1cm, right=1cm]{geometry}
% change margins as you please
% can put for example, paperheight=8.5in,paperwidth=5.5in to get folded letter paper, or other paper sizes

%calculation and graphing packages
\usepackage{calc, tikz, ifthen}
\usepackage[nomessages]{fp}
\usetikzlibrary{calc, intersections}
%
%
% Basic input 
%
\makeatletter
\def\scriptname{Carolingian} % just used for the title which you can get rid of
\def\unit{mm} % unit for everything, supports mm, in, cm, pt, and probably other stuff
\def\nibwidthwithoutunit{2} % nib width in unit, for pointed pen just enter x-height and fill in proportions below, or put 1 in nibwidth and put the ascenders, etc in number of units below
\def\nibwidth{\nibwidthwithoutunit \unit}
\def\ascender{4} % ascender in nib widths
\def\xheight{2.5} % x-height in nib widths
\def\descender{3} % descender in nib widths
\def\linespace{2} % space between lines in nib widths
% slant
\newboolean{ifslant}
\setboolean{ifslant}{true} % change to false if not needed
\def\slant{85} % slant from horizontal. eg. spencerian would be 52 and italics would be in the 80s
\def\slantfreq{10} % length between each slant line in unit (if you don't see slant lines it's probably because you used cm / in and didn't change this)
%
%
% Advanced options
%
% optional short ascender / descender
\newboolean{ifshortascender}
\setboolean{ifshortascender}{true} % change to false if not needed
\def\shortascender{2}
\newboolean{ifshortdescender}
\setboolean{ifshortdescender}{true} % change to false if not needed
\def\shortdescender{1}
% nib ladder
\newboolean{ifladder} % boolean for nib ladder
\setboolean{ifladder}{true} % change to false if not needed
\newboolean{ifladderall} % boolean for whether nib ladder is shown for all lines or just the first line
\setboolean{ifladderall}{true} % change to false if you only want nib ladder for the first line
%
%
% Cosmetics
%
% Default horizontal lines (the ones between ascender, x-height, descender, and line space)
\def\lineweight{0.2pt} % line weight, adjust for thicker / thinner lines
\def\linepattern{solid} % options: solid, dashed, dotted, dashdotted, densely dotted, loosely dotted, double, or you can even write something like
% dash pattern=on 3mm off 1mm on 2mm off 0.5mm 
% to make your own pattern
\definecolor{linecolour}{RGB}{0,0,0} % enter colour code for the lines
% you can always expand the loop below to get different settings for each line.
%
% cosmetics for slant, see above for explanation
\def\slantweight{0.2pt}
\def\slantpattern{solid}
\definecolor{slantcolour}{RGB}{0,0,0}
%
% cosmetics for the short ascenders and descenders line
\def\shortlinesweight{0.1pt}
\def\shortlinespattern{dotted}
\definecolor{shortlinescolour}{RGB}{0,0,0}
%
% fill line space, decide if you want the space between lines filled by a different colour
\newboolean{iffill}
\setboolean{iffill}{true} % change to false if not needed
\definecolor{fillcolour}{RGB}{200,200,200} % enter colour code
%
% cosmetics for nib ladder
\definecolor{laddercolour}{RGB}{0,0,0} % enter colour code for colour of the nib ladder
%
%
% converting units, ignore
\def\convertunits#1#2{\strip@pt\dimexpr #2*65536/\number\dimexpr 1#1}
\def\pagewidth{\convertunits{\unit}{\textwidth}}
\def\pagelength{\convertunits{\unit}{\textheight}}
\def\titleheight{\convertunits{\unit}{\baselineskip}}
\makeatother
\begin{document}
	% calculations, ignore--I just couldn't figure out how to do this within the \draw command
	\FPeval{\ascx}{\ascender + \xheight}
	\FPeval{\letterh}{\ascx+\descender}
	\FPeval{\lineheight}{\letterh+\linespace}
	\FPeval{\lineheightinunits}{\lineheight * \nibwidthwithoutunit}
	\FPadd{\pagewidth}{\pagewidth}{0} % for some reason I need this or it says UPN stack empty
	\FPsub{\pagelength}{\pagelength}{\titleheight} % need to subtract height of title
	% change \titleheight if you're using a larger title
	\FPeval{\lines}{trunc(\pagelength/ \lineheightinunits:0)}
	\FPeval{\actuallength}{\lines * \lineheight * \nibwidthwithoutunit}
	%
	% short ascender / descender calculations
	\ifthenelse{\boolean{ifshortascender}}{
		\FPeval{\shortasc}{\ascender - \shortascender}}{}
	\ifthenelse{\boolean{ifshortdescender}}{
		\FPeval{\shortdesc}{\ascx + \shortdescender}}{}
	%
	% slant calculations
	\ifthenelse{\boolean{ifslant}}{
		\FPeval{\slantradians}{(\FPpi*\slant) / \pagewidth}
		\FPeval{\slantbegin}{\pagelength / tan \slantradians}
		\FPeval{\slantend}{(\slantbegin+\pagewidth) / \slantfreq}
		\FPeval{\slantlength}{(\pagelength^2+\pagewidth^2)^(1/2)}
	}{}
	%
	% gets rid of page numbering because I love everything else about the article class
	\pagenumbering{arabic}
	%
	% title line
	\centering \scriptname\space \ascender-\xheight-\descender, nib width \nibwidth \\
	% insert whatever title you want, or delete this line if you want a clean page of guidelines
	%
	% actual plotting
	\begin{tikzpicture}[x=1\unit, y=1\unit]
	% clip to reduce mess
	\coordinate (A) at (0, 0);
	\coordinate (B) at (\pagewidth, 0);
	\coordinate (C) at (0, -\actuallength);
	\coordinate (D) at (\pagewidth, -\actuallength);
	\clip (A) -- (B) -- (D) -- (C) -- (A);
	%
	% slant lines, will not show if ifslant is set to false
	\ifthenelse{\boolean{ifslant}}{
		\foreach \x in {0,...,\slantend}{
			\pgfmathsetmacro\slantshift{\x * \slantfreq \unit}
			\draw [line width=\slantweight, \slantpattern, draw=slantcolour, xshift=\slantshift] (-\slantbegin, -\pagelength) -- +(\slant:\slantlength);}
	}{}
	%
	% horizontal lines
	\foreach \x in {0,...,\lines}{
		\pgfmathsetmacro\result{-\x * \lineheight * \nibwidth}
		\ifthenelse{\boolean{iffill}}{
			\fill[fill=fillcolour, yshift=\result] (0,-\letterh*\nibwidth) rectangle (\pagewidth,-\lineheight*\nibwidth);
		}{}
		\foreach \length in {0, \ascender, \ascx, \letterh, \lineheight}{
			\draw [line width=\lineweight, \linepattern, draw=linecolour, yshift=\result] (0, -\length*\nibwidth ) -- +(\pagewidth, 0);
			% if you want different line settings for different lines just expand the foreach loop
			% if you're reading through this you probably don't need instructions but just in case:
			%    \foreach \thing in {\something, \someotherthing}{
			%        \draw[pattern] (0, \thing)--(1, \thing)
			%    }
			% can be changed to 
			%    \draw[pattern] (0, \something)--(1, \something)
			%    \draw[pattern] (0, \someotherthing)--(1, \someotherthing)
			% and now you can put different patterns for each line.
		}
		%
		% short ascenders and descenders, will not show if ifshortascender and ifshortdescender are set to false
		\ifthenelse{\boolean{ifshortascender}}{
			\draw [line width=\shortlinesweight, \shortlinespattern, draw=shortlinescolour, yshift=\result] (0, -\shortasc*\nibwidth ) -- +(\pagewidth, 0);}{}
		\ifthenelse{\boolean{ifshortdescender}}{
			\draw [line width=\shortlinesweight, \shortlinespattern, draw=shortlinescolour, yshift=\result] (0, -\shortdesc*\nibwidth ) -- +(\pagewidth, 0);}{}
		%
		% nib ladder, will not show if ifladder is set to false
		\ifthenelse{\boolean{ifladder}}{
			\begin{scope}
			\ifthenelse{\boolean{ifladderall}}{
				\pgfmathsetmacro\laddershift{\result}
			}
			{\pgfmathsetmacro\laddershift{0}} 
			% quite a bit of extra work when it's just one instance, but oh well, the compilation speed is still decent
			% please don't judge
			\coordinate[yshift=\laddershift] (E) at (0, 0);
			\coordinate[yshift=\laddershift] (F) at (2*\nibwidth, 0);
			\coordinate[yshift=\laddershift] (G) at (2*\nibwidth, -\letterh * \nibwidth);
			\coordinate[yshift=\laddershift] (H) at (0, -\letterh * \nibwidth);
			\clip (E) -- (F) -- (G) -- (H) -- (E);
			\foreach \len in {0, 2,...,\letterh}{
				\pgfmathsetmacro\shift{\laddershift - \len*\nibwidth}
				\draw[draw=none, fill=laddercolour, yshift=\shift] (0,0) rectangle (\nibwidth,-\nibwidth);
				\draw[draw=none, fill=laddercolour, yshift=\shift] (\nibwidth,-\nibwidth) rectangle (2*\nibwidth,-2*\nibwidth);
			}
			\end{scope}
		}{}
		
	}
	\end{tikzpicture}
\end{document}
